
%%%%%%%%%%%%%%%%%%%%%%%%%%%%%%%%
%%%%%%%%%%%% PACKAGES %%%%%%%%%%
%%%%%%%%%%%%%%%%%%%%%%%%%%%%%%%%
\usepackage{psfrag}
\usepackage{amsmath}
\usepackage{lipsum}
\usepackage{mathtools}
\usepackage{ifthen}
\usepackage{verbatim}
\usepackage{calc}
%\usepackage{multimedia}
\usepackage{color}
\usepackage{colortbl}
\usepackage{graphics,graphicx,amssymb}
\usepackage{amsxtra}
\usepackage{epsfig}
\usepackage{subfigure}
%\usepackage{movie15}
%\usepackage{media9}
\usepackage{multimedia}
\usepackage{pdfpages}
\usepackage{pgfpages}
\usepackage{tikz}
\usetikzlibrary{arrows,automata}
\usetikzlibrary{mindmap,trees,backgrounds}
\usepackage[crop=pdfcrop]{pstool}
\usepackage{pgflibraryshapes}
%\usepackage{beamerthemesplit}
\usepackage{wrapfig}
\usepackage{mdframed}
\usepackage{setspace}
\usepackage{tcolorbox}
\usepackage[normalem]{ulem}
\usepackage[absolute, overlay]{textpos}
\usepackage{ulem}
%\usepackage{wasysym}
\usepackage{multirow}
\usepackage{tcolorbox}
%%%%%%%%%%%%%%%%%%%%%%%%%%%%%%%%
%%%%%%%%%%%% MACROS %%%%%%%%%%%%%%
%%%%%%%%%%%%%%%%%%%%%%%%%%%%%%%%

%\usepackage{etoolbox}
\setbeamertemplate{theorems}[numbered]
\undef{\lemma}
\undef{\example}
\undef{\proposition}
\undef{\corollary}
\undef{\assumption}
\undef{\definition}
\undef{\conjecture}
\newtheorem{lemma}{\translate{Lemma}}
\newtheorem{proposition}{\translate{Proposition}}
%\theoremstyle{example}
\newtheorem{example}{\translate{Example}}
\newtheorem{corollary}{\translate{Corollary}}
\newtheorem{assumption}{\translate{Assumption}}
\newtheorem{definition}{\translate{Definition}}
\newtheorem{conjecture}{\translate{Conjecture}}

%\setbeamertemplate{theorem}[ams style]
%\setbeamertemplate{theorems}[numbered]
%%
%\newtheorem{helpproposition}[theorem]{Proposition}     
%\newenvironment{proposition}     
%{\vskip.1cm\begin{helpproposition}\it}     
%{\end{helpproposition}\vskip.1cm}     
%%
%\newtheorem{helpexample}[theorem]{Example}     
%\renewenvironment{example}     
%{\vskip.1cm\begin{helpexample}\it}     
%{\end{helpexample}\vskip.1cm}     
%%
%\newtheorem{helpassumption}[theorem]{Assumption}     
%\newenvironment{assumption}     
%{\vskip.1cm\begin{helpassumption}\it}     
%{\end{helpassumption}\vskip.1cm}     

%\setbeamertemplate{proposition}[ams style]
%\setbeamertemplate{propositions}[numbered]
%\setbeamertemplate{example}[ams style]
%\setbeamertemplate{examples}[numbered]

%\newenvironment{example*}
%  {\addtocounter{theorem}{-1}\example}
%  {\endexample}
%\newenvironment{proposition}
%  {\addtocounter{theorem}{-1}\proposition}
%  {\endexample}
%\newenvironment{assumption*}
%  {\addtocounter{theorem}{-1}\assumption}
%  {\endexample}


%\newtheorem{helptheorem}{Theorem}[section]     
%\newtheorem{helpproposition}[helptheorem]{Proposition}     
%\newenvironment{proposition}     
%{\vskip.1cm\begin{helpproposition}\it}     
%{\end{helpproposition}\vskip.1cm}     
%\newtheorem{helpassumption}[helptheorem]{Assumption}
%\newenvironment{assumption}
%{\vskip.1cm\begin{helpassumption}}     
%{\end{helpassumption}\vskip.1cm}


\usepackage{rgsMacros}
\usepackage{rgsBeamerv01}
%\definecolor{mygreen}{RGB}{0,128,0}

\renewcommand{\em}{\it \color{blue}}

\definecolor{lightblue}{RGB}{60,60,200}
\definecolor{mygreen}{RGB}{0,128,0}
\definecolor{catch}{RGB}{0,128,0}
\definecolor{throw}{RGB}{0,0,255}
\definecolor{recovery}{RGB}{255,0,0}
\definecolor{applegreen}{rgb}{0.55, 0.71, 0.0}

\setbeamercolor{block title}{bg= lightblue!10!white}
\setbeamercolor{block body}{bg= lightblue!5!white}
\setbeamercolor{block title alerted}{bg=blue!10!white}
\setbeamercolor{block body alerted}{bg=blue!3!white}
\setbeamerfont{block title}{size={}}


\renewcommand{\comment}[1]{\vspace{1in}{\color{red}\noindent \tt \tiny #1}}
\newcommand{\BAn}[1]{{\cal B}_{#1}}
\newcommand{\B}{{\cal B}}
\renewcommand{\S}{{\cal S}}
\newcommand{\E}{{\cal E}}
\renewcommand{\P}{{\cal P}}
% \renewcommand{\Q}{{\cal Q}}
% \renewcommand{\K}{{\cal K}}

\newcommand{\Ir}{{\cal I}(r)}
\newcommand{\Irzero}{{\cal I}(0)}

\newcommand{\putat}[3]{\begin{picture}(0,0)(0,0)\put(#1,#2){#3}\end{picture}} % just a shorthand
\renewcommand{\U}{{\cal U}}
% \newcommand{\V}{{\cal V}}

\newcommand{\tb}{T}
\newcommand{\ton}{x_1}
\newcommand{\ttw}{x_2}

\newcommand{\light}[1]{\textcolor{gray}{#1}}
\newcommand{\Id}{I}
\newcommand{\ep}{{\varepsilon}}
\newcommand{\Tc}{T_u}
\newcommand{\Ts}{T_s}
\newcommand{\tauc}{\tau_u}
\newcommand{\taus}{\tau_s}
\newcommand{\lambdac}{\lambda_u}
\newcommand{\lambdas}{\lambda_s}
%\newcommand{\V}{{\reals^{n_P}}}
%\renewcommand{\K}{{\cal K}}
\newcommand{\diag}{\mathop{\mbox{diag}}\nolimits}
\newcommand{\he}{\mathop{\mbox{He}}\nolimits}
\newcommand{\wt}{\widetilde}
\renewcommand{\T}{\mathop{\cal T}\nolimits}
%\newcommand{\W}{\mathcal{W}}
%%%% FOR TIKZ PICTURE

%\usepackage{tikz}
%\usetikzlibrary{mindmap,trees}
%
%\tikzset{level 1 concept/.append style={font=\sf, sibling angle=90,level distance = 27mm}}
%\tikzset{level 2 concept/.append style={font=\sf, sibling angle=45,level distance = 17mm}}
%%\tikzset{every node/.append style={scale=1}}    
%\tikzset{every node/.append style={scale=0.6}}    
%
%%% ADDED TO MATCH MARCELLO's TKIZ command
%\tikzstyle{block} = [draw, fill=blue!20, rectangle, 
%    minimum height=3em, minimum width=6em]
%\tikzstyle{sum} = [draw, fill=blue!20, circle, node distance=1cm]
%\tikzstyle{input} = [coordinate]
%\tikzstyle{output} = [coordinate]
%\tikzstyle{pinstyle} = [pin edge={to-,thin,black}]



%%%%%%%%%%%%%%%%%%%%%%%%%%%%%%%%
%%%%%%%%%%%% SETTINGS %%%%%%%%%%%%%%
%%%%%%%%%%%%%%%%%%%%%%%%%%%%%%%%

\usetheme{rgsv02nopic}
%\pgfpagesuselayout{4 on 1}[letterpaper,border shrink=5mm] 


%\graphicspath{{Figures/}}

%\makeatletter
%\def\input@path{{/Users/Ricardo/GoogleDrive/PyTexLibrary/WorkshopCDC19/BerkAltin/Figures/}}
%\makeatother

\def\cmd#1{\texttt{\textbackslash #1}}
\def\env#1{\texttt{#1}}

%%%%%%%%%%% TableOfContents Style %%%%%%%%%%%%%%%%%%

\setbeamertemplate{section in toc}{%
	\bf \large {\color{red} \inserttocsectionnumber.}~\inserttocsection}
\setbeamercolor{subsection in toc}{bg=white,fg=structure}
\setbeamertemplate{subsection in toc}{%
	\vspace{3mm} \hspace{5mm} {\color{applegreen} $\blacktriangleright$}~\inserttocsubsection\par}

%\setbeamertemplate{navigation symbols}{}
\setwhitesheetcolors

% COMMENT OUT TO SHOW LABELS
%\usepackage[inline]{showlabels}

% Rafal commands
\newcommand{\Sol}{{\cal S}}
\newcommand{\length}{\mathop{\rm length}}
\newcommand{\Length}{\mathop{\rm Length}}
\newcommand{\ve}{\varepsilon}
\newcommand{\ds}{\displaystyle} 
\newcommand{\BPA}{\mathcal{B}(\A)}
\newcommand{\sola}{\psi}
\newcommand{\limtj}{\lim_{t+j\to\infty}}
\newcommand{\limtk}{\lim_{t+k\to\infty}}
% \renewcommand{\sol}{\phi}

%% Hybrid control commands

%% plant system
\newcommand{\HSp}{\HS_P}
%% plant's state
\newcommand{\xp}{z}
\newcommand{\zp}{z_p}
\newcommand{\zpdot}{\dot{z}_p}
%%
\newcommand{\xpdot}{\dot{\xp}}
%%
\newcommand{\xpddot}{\ddot{\xp}}
%% plant's input
\newcommand{\up}{u}
%% plant's output
\newcommand{\yp}{y}
%% plant's right-hand side
\newcommand{\fp}{F_P}
\newcommand{\gp}{G_P}
\newcommand{\Cp}{C_P}
\newcommand{\olCp}{\overline{C}_P}
\newcommand{\Dp}{D_P}
%% plant's output function
\newcommand{\hp}{h}
%% plant's input space
\newcommand{\upSpace}{{\cal U}_P}
%% plant's state space
\newcommand{\xpSpace}{{\cal X}_P}
%% plant's output space
\newcommand{\ypSpace}{{\cal Y}_P}
%% controller system
% \renewcommand{\HSc}{\HS_K}
%% controller's state
\newcommand{\xc}{\eta}
%%
\newcommand{\xcdot}{\dot{\xc}}
%%
\newcommand{\xcddot}{\ddot{\xc}}
%% controller's input
\newcommand{\uc}{v}
%% controller's output
\newcommand{\yc}{\zeta}
%% controller's right-hand side
\newcommand{\fc}{F_K}
\newcommand{\gc}{G_K}
%% controller's output function
\newcommand{\hc}{\kappa}
\newcommand{\Cc}{C_K}
\newcommand{\Dc}{D_K}
%% controller's input space
\newcommand{\ucSpace}{{\cal U}_K}
%% controller's state space
\newcommand{\xcSpace}{{\cal X}_K}
%% logic state
\newcommand{\xlogic}{q}
\newcommand{\xlogicdot}{\dot{\xlogic}}
\newcommand{\xlogicSpace}{Q}
%\newcommand{\xcSpace}{\Upsilon}
%\newcommand{\ucSpace}{{\cal V}}

%% timer state
\newcommand{\xtimer}{\tau}
\newcommand{\xtimerdot}{\dot{\xtimer}}
\newcommand{\xtimerp}{T^*}
\newcommand{\xtimerSpace}{[0,\xtimerp]}

%% memory state
\newcommand{\xmem}{\ell}
\newcommand{\xmemdot}{\dot{\xmem}}

%$ measurement noise
\newcommand{\m}{m}

%% closed loop's state
\newcommand{\x}{x}
% \renewcommand{\xdot}{\dot{x}}
%% closed-loop's right-hand side
\newcommand{\f}{F}
% \newcommand{\g}{G}
\renewcommand{\C}{C}
\newcommand{\olC}{\overline{C}}
\renewcommand{\D}{D}
%% plant's state space
\newcommand{\xSpace}{{\cal X}}
% \newcommand{\h}{h}


%%Logic-based control
\newcommand{\HScq}{\HS_{K,q}}
\newcommand{\xcq}{\xc_q}
\newcommand{\xcdotq}{\dot{\xc}_q}
%% controller's input
\newcommand{\ucq}{v_q}
%% controller's output
\newcommand{\ycq}{\zeta_q}
%% controller's right-hand side
\newcommand{\fcq}{F_{K,q}}
\newcommand{\gcq}{G_{K,q}}
%% controller's output function
\newcommand{\hcq}{\kappa_q}
\newcommand{\Ccq}{C_{K,q}}
\newcommand{\Dcq}{D_{K,q}}
\newcommand{\ucSpaceq}{{\cal U}_{K,q}}
%% controller's state space
\newcommand{\xcSpaceq}{{\cal X}_{K,q}}

%Supervisor System
\newcommand{\HSs}{\HS_{S}}
\newcommand{\Cs}{C_S}
\newcommand{\Ds}{D_S}
\newcommand{\Csq}{C_{S,q}}
\newcommand{\Dsq}{D_{S,q}}
\newcommand{\gs}{G_{S}}
\newcommand{\us}{v_{S}}

% Jun's command
\definecolor{lightred}{RGB}{255,100,100}% red
\definecolor{mediumorchid}{RGB}{186,85,211} % purple
\definecolor{orange}{RGB}{255,165,0}% orange
%\newcommand*{\hlw}[1]{
%	\tikz[baseline=(X.base)] \node[rectangle, fill=white] (X) {#1};
%}
%\newcommand{\ifeq}{\text{\normalfont if }}
%\newcommand{\jref}[1]{{\color{lightblue}{#1}}}
%\newcommand{\chai}[1]{{\color{mediumorchid}{#1}}}
%\newcommand{\jun}[1]{{\color{orange}{#1}}}
%\newcommand{\JC}[1]{{\color{mygreen}{#1}}}
%\newcommand{\FI}[1]{{\color{mediumorchid}{#1}}}
%\newcommand{\flow}[1]{{\color{blue}{#1}}}
%\newcommand{\jump}[1]{{\color{red}{#1}}}
%\newcommand{\otherw}{\text{\normalfont otherwise}}
\newcommand{\sg}{\text{\normalfont\scriptsize g}}
\newcommand{\sfw}{\text{\normalfont\scriptsize fw}}
\newcommand{\sS}{\text{\normalfont\scriptsize s}}
\renewcommand{\M}{{\cal M}_r} % comflict with RGS command
\newcommand{\Lc}{\Theta_c}
\newcommand{\Ld}{\Theta_d}
\newcommand{\wLc}{\widetilde{\Theta}_c}
\newcommand{\wLd}{\widetilde{\Theta}_d}
\newcommand{\rc}{\rho_c}
\newcommand{\rd}{\rho_d}
\newcommand{\Lstar}{\Theta_\star}
\newcommand{\wLstar}{\widetilde{\Theta}_\star}
\newcommand{\mc}{\reals^n \times {\cal U}_c}
\newcommand{\md}{\reals^n \times {\cal U}_d}
%\newcommand{\inter}{\text{\normalfont int }}
\newcommand{\Hcl}{\HS}%{\HS_{cl}}
\newcommand{\Ccl}{C}%{C_{cl}}
\newcommand{\Fcl}{F}%{F_{cl}}
\newcommand{\Dcl}{D}%{D_{cl}}
\newcommand{\Gcl}{G}%{G_{cl}}

%%Passivity Talk Commands
\def\ba{\begin{array}}
\def\ea{\end{array}}
\def\be{\begin{equation}}
\def\ee{\end{equation}}
\def\fraz{\dst\frac}
\def\qmxr#1{\left (\begin{matrix} #1\end{matrix}\right )}
\def\qmxq#1{\left [\begin{matrix} #1\end{matrix}\right ]}
\def\qmx#1{\left (\begin{matrix} #1\end{matrix}\right )}
\def\parder#1_#2{\fraz{\partial #1}{\partial #2}}
\def\pardertr#1_#2{\fraz{\partial\tr #1}{\partial #2}}
\def\unmezzo{\fraz{1}{2}}
%%
\def\HSZero{\mathcal{H}_0}
\def\u{v}

\def\uflow{u_c}
\def\uflowTwo{\tilde{u}_c}
\def\uflowmap{u_c}
\def\uflowset{u_c}

\def\ujump{u_d}
\def\ujumpTwo{\tilde{u}_d}
\def\ujumpmap{u_d}
\def\ujumpset{u_d}

\def\yflow{y_c}
\def\yflowmap{y}
\def\yflowset{y}

\def\yjump{y_d}
\def\yjumpmap{y}
\def\yjumpset{y}

\def\restcoefficent{\varrho}
\def\restcoefficentBall{e_c}
%OLD
\def\uFlowSupply{u^\top \Gamma_c^\top}
\def\uJumpSupply{u^\top \Gamma_d^\top}

\def\yFlowSupply{\Phi_c h(x,\u)}
\def\yJumpSupply{\Phi_d h(x,\u)}
\def\SupplyOutputFlow{y^\top \Phi_c^\top \Phi_c \rho_c(y)}
\def\SupplyOutputJump{y^\top \Phi_d^\top \Phi_d \rho_d(y)}
\def\SupplyOutputFlowZeroInput{h(x,0)^\top \Phi_c^\top \Phi_c \rho_c(h(x,0))}
\def\SupplyOutputJumpZeroInput{h(x,0)^\top \Phi_d^\top \Phi_d \rho_d(h(x,0))}


%POSSIBLE NEW:
\def\uFlowSupply{u_c^\top}
\def\uJumpSupply{u_d^\top}
\def\yFlowSupply{y_c}
\def\yJumpSupply{y_d}
\def\SupplyOutputFlow{y_c^\top \rho_c(y_c)}
\def\SupplyOutputJump{y_d^\top \rho_d(y_d)}
\def\SupplyOutputFlowZeroInput{h_c(x,0)^\top \rho_c(h_c(x,0))}
\def\SupplyOutputJumpZeroInput{h_d(x,0)^\top \rho_d(h_d(x,0))}




\def\virtualInputTwo{\tilde{u}_c}
\def\virtualInputThree{\hat{u}_c}


%old F_c
\def\FCompliant{f_{c}}


%passivity based control:

\def\kappac{\kappa_c}
\def\kappad{\kappa_d}

\def\startmodif{\color{magenta}}
\def\stopmodif{\color{black}\normalcolor}
\def\ricardo#1{\color{blue}#1\normalcolor}
\def\roberto#1{\color{red}#1\normalcolor}

\DeclareMathOperator*{\argmin}{arg\,min}

% FROM JUN

\newcommand{\flow}[1]{{\color{blue}{#1}}}
\newcommand{\jump}[1]{{\color{red}{#1}}}

\newcommand{\FI}[1]{{\color{mygreen}{#1}}}
\newcommand{\chai}[1]{{\color{mygreen}{#1}}}
\newcommand{\JC}[1]{{\color{lightblue}{#1}}}
\newcommand{\jun}[1]{{\color{purple}{#1}}}

\newcommand{\ifeq}{\text{\normalfont if }}
\newcommand{\otherw}{\text{\normalfont otherwise}}
\newcommand{\inter}{\text{\normalfont int}}

\newcommand{\data}{(C,F,D,G)}
\newcommand{\z}{\x}
\newcommand{\hu}{\hat{u}}
\newcommand{\hy}{\hat{y}}
\newcommand{\hF}{\hat{F}}

\newcommand{\Nset}{\ricardo{\{1, 2, ... , N\}}}
\newcommand{\Nsety}{\{1, 2, ... , N_y\}}
\newcommand{\Nsetu}{\{1, 2, ... , N_u\}}
\newcommand{\iy}{{i_y}}
\newcommand{\iu}{{i_u}}
\newcommand{\ET}{\gamma} %  triggering event symble
\newcommand{\ETfy}{\gamma^y_\iy} % output triggering event
\newcommand{\ETfu}{\gamma^u_\iu} % input triggering event
\newcommand{\ETarg}{\xi}
\newcommand{\gfy}{g^y_{\iy}}
\newcommand{\gfu}{g^u_{\iu}}
\newcommand{\jy}{j_y}
\newcommand{\ju}{j_u}

%letters
\newcommand{\Cmc}{\mathcal{C}}
\newcommand{\Dmc}{\mathcal{D}}
\newcommand{\Lmc}{\mathcal{L}}
\newcommand{\Bmc}{\mathcal{B}}
\newcommand{\Xmc}{\mathcal{X}}

%%%%%%%%%%%%%%%%%%%%%%%%%%%%%%%%%%%%%%%%%%%%%%%%%%
% Berk's stuff
%\usepackage{tikz,circuitikz,tikz-cd,tikzscale}
%\usetikzlibrary{shapes,arrows,calc,fit}

%\usepackage{animate}

%\usepackage{pgfplots}
%\pgfplotsset{compat=newest}
%\usepgfplotslibrary{fillbetween}

%\usepgfplotslibrary{external}
%\tikzexternalize[prefix=tikz/]	%create folder named tikz

%\def\Put(#1,#2)#3{\leavevmode\makebox(0,0){\put(#1,#2){#3}}}


%Define tikz shapes
%\tikzstyle{blocknone} = [draw=none,rectangle,minimum height=1.75em,minimum width=6em]
%\tikzstyle{block} = [draw,rectangle,minimum height=1.75em,minimum width=6em]
%\tikzstyle{systemblock} = [draw,rectangle,minimum height=1.75em,minimum width=3em]
%\tikzstyle{flowblock} = [draw,rectangle,rounded corners,minimum height=3em,minimum width=20em]
%\tikzstyle{gain} = [draw,rectangle,minimum height=1em,minimum width=1em]
%\tikzstyle{sum} = [draw,circle,node distance=0.5cm]
%\tikzstyle{signal} = [coordinate]
%\tikzstyle{pinstyle} = [pin edge={to-,thin,black}]
%
%\newcommand{\Anorm}[1]{\left|#1\right|_{\cal A}}
%
%\usepackage{algorithm,algpseudocode} % \usepackage for algorithm
%
%\newcommand{\flowcost}{L_c}
%\newcommand{\jumpcost}{L_d}
%%\newcommand{\kappac}{\kappa_c}
%%\newcommand{\kappad}{\kappa_d}
%\newcommand{\alphac}{\alpha_c}
%\newcommand{\alphad}{\alpha_d}
%\newcommand{\gammac}{\gamma_c}
%\newcommand{\gammad}{\gamma_d}
%\newcommand{\totalenergy}{W}
%%%%%%%%%%%%%%%%%%%%%%%%%%%%%%%%%%%%%%%%%%%%%%%%%%

%%%%%%%%%%%%%%%%%%%%%%%%%%%%%%%%%%%%%%%%%%%%%%%%%%
% Parmita's stuff
\usepackage{calc}

\usepackage{listings}

\definecolor{mygreen}{RGB}{28,172,0} % color values Red, Green, Blue

\definecolor{mylilas}{RGB}{170,55,241}

%\lstset{language=Matlab,
%    %basicstyle=\color{red},
%    breaklines=true,
%    morekeywords={matlab2tikz},
%    keywordstyle=\color{blue},
%    morekeywords=[2]{1}, keywordstyle=[2]{\color{black}},
%    identifierstyle=\color{black},
%    stringstyle=\color{mylilas},
%    commentstyle=\color{mygreen},
%    showstringspaces=false,
%    numbers=left,
%    numberstyle={\tiny \color{black}},
%    numbersep=9pt,
%    emph=[1]{for,end,break},emphstyle=[1]\color{red},
%}

%% FROM MOHAMED
%\usepackage{bm}
%\newcommand\red[1]{{\color{red}#1}}
%\newcommand{\bsym}[1]{\boldsymbol{#1}}
%\newcommand{\mrm}[1]{\mathrm{#1}}
%\renewcommand{\mc}[1]{\mathcal{#1}}
%\newcommand{\mbb}[1]{\mathbb{#1}}
%\newcommand{\munit}[1]{[\mathrm{#1}]}
%
%% FOR HYBRID MPC
%
%\newcommand{\F}{{\cal F}}     
%\renewcommand{\L}{{\cal L}}     
%\newcommand*{\tr}{^{\mkern-1.5mu\mathsf{T}}}
%
%%Francesco's stuff
%\newcommand{\Js}{\mathcal{J}^\star}
%\newcommand{\xh}{\hat{x}}
%\newcommand{\He}{\operatorname{He}}
%\newcommand*{\QEDB}{\hfill\ensuremath{\square}}%
%\newcommand{\Sn}{\mathbb{S}^n} 
%\newcommand{\SHO}{\mathcal{S}_{\HS}}
%\newcommand{\CUD}{\overline{\C}\cup \D}
%\newcommand{\limdom}{\lim_{\substack{(t, j)\in\dom\x\\(t, j)\rightarrow\sup\dom\x}}}
%\newcommand{\limdomhat}{\lim_{\substack{(t, j)\in\dom\xh\\(t, j)\rightarrow\sup\dom\xh}}}
%\newcommand{\Cone}{\mathbf{Cone}}

%%%%%%%%%%%%%%%%%%%%%%%%%%%%%%%%%%%%%%%%%%%%%%%%%%%%%%%%%%%%%%%%%%%%%%
%%% VERSION CONTROL COMMAND
%%% For Conference, Change to "true"
%%% For Report, Change to "false"
\usepackage{ifthen}


\newboolean{Inclusion}
\setboolean{Inclusion}{false}
%\newcommand{\IfInc}[2]{\ifthenelse{\boolean{Inclusion}}{ #1 #2}{#2}}
%\newcommand{\IfIncd}[2]{\ifthenelse{\boolean{Inclusion}}{{#1}}{#2}}
\newcommand{\IfInc}[2]{\ifthenelse{\boolean{Inclusion}}{ \color{cyan}#1\color{black}#2}{#2}}
\newcommand{\IfIncd}[2]{\ifthenelse{\boolean{Inclusion}}{{\color{cyan}#1}}{#2}}

\newboolean{Personal}
\setboolean{Personal}{false}
\newcommand{\IfPers}[1]{\ifthenelse{\boolean{Personal}}{{\color{purple}#1} }{}}


\newboolean{Infinite-horizon} 
\setboolean{Infinite-horizon}{true} %False:General framework. True: Only Infinite-horizon.
\newcommand{\IfIh}[2]{\ifthenelse{\boolean{Infinite-horizon}}{#1}{{\color{purple}#2}}}

\newboolean{Two-Players} 
\setboolean{Two-Players}{true} %False:General framework. True: Only Two players formulation.
\newcommand{\IfTp}[2]{\ifthenelse{\boolean{Infinite-horizon}}{#1}{{\color{purple}#2}}}
%%%%%%%%%%%%%%%%%%%%%%%%%%



\tcbuselibrary{skins}

\newtcolorbox{mybox}[1][]{before=\centering, hbox, drop fuzzy shadow, enhanced, colback=white, rounded corners, colframe=lightblue, fonttitle=\bfseries, title=#1, center title}