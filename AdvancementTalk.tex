\pdfminorversion=6
\pdfsuppresswarningpagegroup=1
% PRESENTATION 
\documentclass[aspectratio=169]{beamer}
\errorcontextlines=9999
% TO MAKE HANDOUT use
%\documentclass[notes=hide,handout]{beamer}
%\pgfpagesuselayout{4 on 1}[letterpaper,border shrink=5mm] 

\definecolor{lightGray}{RGB}{230,230,230}

\newcommand{\ExtraW}{20}
\newcommand{\RaiseT}{15}

% LOG
% 08/17: added \usepackage{multirow}

%%%%%%%%%%%%%%%%%%%%%%%%%%%%%%%%
%%%%%%%%%%%% PACKAGES %%%%%%%%%%
%%%%%%%%%%%%%%%%%%%%%%%%%%%%%%%%
\usepackage{psfrag}
\usepackage{amsmath}
\usepackage{lipsum}
\usepackage{mathtools}
\usepackage{ifthen}
\usepackage{verbatim}
\usepackage{calc}
%\usepackage{multimedia}
\usepackage{color}
\usepackage{colortbl}
\usepackage{graphics,graphicx,amssymb}
\usepackage{amsxtra}
\usepackage{epsfig}
\usepackage{subfigure}
%\usepackage{movie15}
%\usepackage{media9}
\usepackage{multimedia}
\usepackage{pdfpages}
\usepackage{pgfpages}
\usepackage{tikz}
\usetikzlibrary{arrows,automata}
\usetikzlibrary{mindmap,trees,backgrounds}
\usepackage[crop=pdfcrop]{pstool}
\usepackage{pgflibraryshapes}
%\usepackage{beamerthemesplit}
\usepackage{wrapfig}
\usepackage{mdframed}
\usepackage{setspace}
\usepackage{tcolorbox}
\usepackage[normalem]{ulem}
\usepackage[absolute, overlay]{textpos}
\usepackage{ulem}
%\usepackage{wasysym}
\usepackage{multirow}
\usepackage{tcolorbox}
%%%%%%%%%%%%%%%%%%%%%%%%%%%%%%%%
%%%%%%%%%%%% MACROS %%%%%%%%%%%%%%
%%%%%%%%%%%%%%%%%%%%%%%%%%%%%%%%

%\usepackage{etoolbox}
\setbeamertemplate{theorems}[numbered]
\undef{\lemma}
\undef{\example}
\undef{\proposition}
\undef{\corollary}
\undef{\assumption}
\undef{\definition}
\undef{\conjecture}
\newtheorem{lemma}{\translate{Lemma}}
\newtheorem{proposition}{\translate{Proposition}}
%\theoremstyle{example}
\newtheorem{example}{\translate{Example}}
\newtheorem{corollary}{\translate{Corollary}}
\newtheorem{assumption}{\translate{Assumption}}
\newtheorem{definition}{\translate{Definition}}
\newtheorem{conjecture}{\translate{Conjecture}}

%\setbeamertemplate{theorem}[ams style]
%\setbeamertemplate{theorems}[numbered]
%%
%\newtheorem{helpproposition}[theorem]{Proposition}     
%\newenvironment{proposition}     
%{\vskip.1cm\begin{helpproposition}\it}     
%{\end{helpproposition}\vskip.1cm}     
%%
%\newtheorem{helpexample}[theorem]{Example}     
%\renewenvironment{example}     
%{\vskip.1cm\begin{helpexample}\it}     
%{\end{helpexample}\vskip.1cm}     
%%
%\newtheorem{helpassumption}[theorem]{Assumption}     
%\newenvironment{assumption}     
%{\vskip.1cm\begin{helpassumption}\it}     
%{\end{helpassumption}\vskip.1cm}     

%\setbeamertemplate{proposition}[ams style]
%\setbeamertemplate{propositions}[numbered]
%\setbeamertemplate{example}[ams style]
%\setbeamertemplate{examples}[numbered]

%\newenvironment{example*}
%  {\addtocounter{theorem}{-1}\example}
%  {\endexample}
%\newenvironment{proposition}
%  {\addtocounter{theorem}{-1}\proposition}
%  {\endexample}
%\newenvironment{assumption*}
%  {\addtocounter{theorem}{-1}\assumption}
%  {\endexample}


%\newtheorem{helptheorem}{Theorem}[section]     
%\newtheorem{helpproposition}[helptheorem]{Proposition}     
%\newenvironment{proposition}     
%{\vskip.1cm\begin{helpproposition}\it}     
%{\end{helpproposition}\vskip.1cm}     
%\newtheorem{helpassumption}[helptheorem]{Assumption}
%\newenvironment{assumption}
%{\vskip.1cm\begin{helpassumption}}     
%{\end{helpassumption}\vskip.1cm}


\usepackage{rgsMacros}
\usepackage{rgsBeamerv01}
%\definecolor{mygreen}{RGB}{0,128,0}

\renewcommand{\em}{\it \color{blue}}

\definecolor{lightblue}{RGB}{60,60,200}
\definecolor{mygreen}{RGB}{0,128,0}
\definecolor{catch}{RGB}{0,128,0}
\definecolor{throw}{RGB}{0,0,255}
\definecolor{recovery}{RGB}{255,0,0}
\definecolor{applegreen}{rgb}{0.55, 0.71, 0.0}

\setbeamercolor{block title}{bg= lightblue!10!white}
\setbeamercolor{block body}{bg= lightblue!5!white}
\setbeamercolor{block title alerted}{bg=blue!10!white}
\setbeamercolor{block body alerted}{bg=blue!3!white}
\setbeamerfont{block title}{size={}}


\renewcommand{\comment}[1]{\vspace{1in}{\color{red}\noindent \tt \tiny #1}}
\newcommand{\BAn}[1]{{\cal B}_{#1}}
\newcommand{\B}{{\cal B}}
\renewcommand{\S}{{\cal S}}
\newcommand{\E}{{\cal E}}
\renewcommand{\P}{{\cal P}}
% \renewcommand{\Q}{{\cal Q}}
% \renewcommand{\K}{{\cal K}}

\newcommand{\Ir}{{\cal I}(r)}
\newcommand{\Irzero}{{\cal I}(0)}

\newcommand{\putat}[3]{\begin{picture}(0,0)(0,0)\put(#1,#2){#3}\end{picture}} % just a shorthand
\renewcommand{\U}{{\cal U}}
% \newcommand{\V}{{\cal V}}

\newcommand{\tb}{T}
\newcommand{\ton}{x_1}
\newcommand{\ttw}{x_2}

\newcommand{\light}[1]{\textcolor{gray}{#1}}
\newcommand{\Id}{I}
\newcommand{\ep}{{\varepsilon}}
\newcommand{\Tc}{T_u}
\newcommand{\Ts}{T_s}
\newcommand{\tauc}{\tau_u}
\newcommand{\taus}{\tau_s}
\newcommand{\lambdac}{\lambda_u}
\newcommand{\lambdas}{\lambda_s}
%\newcommand{\V}{{\reals^{n_P}}}
%\renewcommand{\K}{{\cal K}}
\newcommand{\diag}{\mathop{\mbox{diag}}\nolimits}
\newcommand{\he}{\mathop{\mbox{He}}\nolimits}
\newcommand{\wt}{\widetilde}
\renewcommand{\T}{\mathop{\cal T}\nolimits}
%\newcommand{\W}{\mathcal{W}}
%%%% FOR TIKZ PICTURE

%\usepackage{tikz}
%\usetikzlibrary{mindmap,trees}
%
%\tikzset{level 1 concept/.append style={font=\sf, sibling angle=90,level distance = 27mm}}
%\tikzset{level 2 concept/.append style={font=\sf, sibling angle=45,level distance = 17mm}}
%%\tikzset{every node/.append style={scale=1}}    
%\tikzset{every node/.append style={scale=0.6}}    
%
%%% ADDED TO MATCH MARCELLO's TKIZ command
%\tikzstyle{block} = [draw, fill=blue!20, rectangle, 
%    minimum height=3em, minimum width=6em]
%\tikzstyle{sum} = [draw, fill=blue!20, circle, node distance=1cm]
%\tikzstyle{input} = [coordinate]
%\tikzstyle{output} = [coordinate]
%\tikzstyle{pinstyle} = [pin edge={to-,thin,black}]



%%%%%%%%%%%%%%%%%%%%%%%%%%%%%%%%
%%%%%%%%%%%% SETTINGS %%%%%%%%%%%%%%
%%%%%%%%%%%%%%%%%%%%%%%%%%%%%%%%

\usetheme{rgsv02nopic}
%\pgfpagesuselayout{4 on 1}[letterpaper,border shrink=5mm] 


%\graphicspath{{Figures/}}

%\makeatletter
%\def\input@path{{/Users/Ricardo/GoogleDrive/PyTexLibrary/WorkshopCDC19/BerkAltin/Figures/}}
%\makeatother

\def\cmd#1{\texttt{\textbackslash #1}}
\def\env#1{\texttt{#1}}

%%%%%%%%%%% TableOfContents Style %%%%%%%%%%%%%%%%%%

\setbeamertemplate{section in toc}{%
	\bf \large {\color{red} \inserttocsectionnumber.}~\inserttocsection}
\setbeamercolor{subsection in toc}{bg=white,fg=structure}
\setbeamertemplate{subsection in toc}{%
	\vspace{3mm} \hspace{5mm} {\color{applegreen} $\blacktriangleright$}~\inserttocsubsection\par}

%\setbeamertemplate{navigation symbols}{}
\setwhitesheetcolors

% COMMENT OUT TO SHOW LABELS
%\usepackage[inline]{showlabels}

% Rafal commands
\newcommand{\Sol}{{\cal S}}
\newcommand{\length}{\mathop{\rm length}}
\newcommand{\Length}{\mathop{\rm Length}}
\newcommand{\ve}{\varepsilon}
\newcommand{\ds}{\displaystyle} 
\newcommand{\BPA}{\mathcal{B}(\A)}
\newcommand{\sola}{\psi}
\newcommand{\limtj}{\lim_{t+j\to\infty}}
\newcommand{\limtk}{\lim_{t+k\to\infty}}
% \renewcommand{\sol}{\phi}

%% Hybrid control commands

%% plant system
\newcommand{\HSp}{\HS_P}
%% plant's state
\newcommand{\xp}{z}
\newcommand{\zp}{z_p}
\newcommand{\zpdot}{\dot{z}_p}
%%
\newcommand{\xpdot}{\dot{\xp}}
%%
\newcommand{\xpddot}{\ddot{\xp}}
%% plant's input
\newcommand{\up}{u}
%% plant's output
\newcommand{\yp}{y}
%% plant's right-hand side
\newcommand{\fp}{F_P}
\newcommand{\gp}{G_P}
\newcommand{\Cp}{C_P}
\newcommand{\olCp}{\overline{C}_P}
\newcommand{\Dp}{D_P}
%% plant's output function
\newcommand{\hp}{h}
%% plant's input space
\newcommand{\upSpace}{{\cal U}_P}
%% plant's state space
\newcommand{\xpSpace}{{\cal X}_P}
%% plant's output space
\newcommand{\ypSpace}{{\cal Y}_P}
%% controller system
% \renewcommand{\HSc}{\HS_K}
%% controller's state
\newcommand{\xc}{\eta}
%%
\newcommand{\xcdot}{\dot{\xc}}
%%
\newcommand{\xcddot}{\ddot{\xc}}
%% controller's input
\newcommand{\uc}{v}
%% controller's output
\newcommand{\yc}{\zeta}
%% controller's right-hand side
\newcommand{\fc}{F_K}
\newcommand{\gc}{G_K}
%% controller's output function
\newcommand{\hc}{\kappa}
\newcommand{\Cc}{C_K}
\newcommand{\Dc}{D_K}
%% controller's input space
\newcommand{\ucSpace}{{\cal U}_K}
%% controller's state space
\newcommand{\xcSpace}{{\cal X}_K}
%% logic state
\newcommand{\xlogic}{q}
\newcommand{\xlogicdot}{\dot{\xlogic}}
\newcommand{\xlogicSpace}{Q}
%\newcommand{\xcSpace}{\Upsilon}
%\newcommand{\ucSpace}{{\cal V}}

%% timer state
\newcommand{\xtimer}{\tau}
\newcommand{\xtimerdot}{\dot{\xtimer}}
\newcommand{\xtimerp}{T^*}
\newcommand{\xtimerSpace}{[0,\xtimerp]}

%% memory state
\newcommand{\xmem}{\ell}
\newcommand{\xmemdot}{\dot{\xmem}}

%$ measurement noise
\newcommand{\m}{m}

%% closed loop's state
\newcommand{\x}{x}
% \renewcommand{\xdot}{\dot{x}}
%% closed-loop's right-hand side
\newcommand{\f}{F}
% \newcommand{\g}{G}
\renewcommand{\C}{C}
\newcommand{\olC}{\overline{C}}
\renewcommand{\D}{D}
%% plant's state space
\newcommand{\xSpace}{{\cal X}}
% \newcommand{\h}{h}


%%Logic-based control
\newcommand{\HScq}{\HS_{K,q}}
\newcommand{\xcq}{\xc_q}
\newcommand{\xcdotq}{\dot{\xc}_q}
%% controller's input
\newcommand{\ucq}{v_q}
%% controller's output
\newcommand{\ycq}{\zeta_q}
%% controller's right-hand side
\newcommand{\fcq}{F_{K,q}}
\newcommand{\gcq}{G_{K,q}}
%% controller's output function
\newcommand{\hcq}{\kappa_q}
\newcommand{\Ccq}{C_{K,q}}
\newcommand{\Dcq}{D_{K,q}}
\newcommand{\ucSpaceq}{{\cal U}_{K,q}}
%% controller's state space
\newcommand{\xcSpaceq}{{\cal X}_{K,q}}

%Supervisor System
\newcommand{\HSs}{\HS_{S}}
\newcommand{\Cs}{C_S}
\newcommand{\Ds}{D_S}
\newcommand{\Csq}{C_{S,q}}
\newcommand{\Dsq}{D_{S,q}}
\newcommand{\gs}{G_{S}}
\newcommand{\us}{v_{S}}

% Jun's command
\definecolor{lightred}{RGB}{255,100,100}% red
\definecolor{mediumorchid}{RGB}{186,85,211} % purple
\definecolor{orange}{RGB}{255,165,0}% orange
%\newcommand*{\hlw}[1]{
%	\tikz[baseline=(X.base)] \node[rectangle, fill=white] (X) {#1};
%}
%\newcommand{\ifeq}{\text{\normalfont if }}
%\newcommand{\jref}[1]{{\color{lightblue}{#1}}}
%\newcommand{\chai}[1]{{\color{mediumorchid}{#1}}}
%\newcommand{\jun}[1]{{\color{orange}{#1}}}
%\newcommand{\JC}[1]{{\color{mygreen}{#1}}}
%\newcommand{\FI}[1]{{\color{mediumorchid}{#1}}}
%\newcommand{\flow}[1]{{\color{blue}{#1}}}
%\newcommand{\jump}[1]{{\color{red}{#1}}}
%\newcommand{\otherw}{\text{\normalfont otherwise}}
\newcommand{\sg}{\text{\normalfont\scriptsize g}}
\newcommand{\sfw}{\text{\normalfont\scriptsize fw}}
\newcommand{\sS}{\text{\normalfont\scriptsize s}}
\renewcommand{\M}{{\cal M}_r} % comflict with RGS command
\newcommand{\Lc}{\Theta_c}
\newcommand{\Ld}{\Theta_d}
\newcommand{\wLc}{\widetilde{\Theta}_c}
\newcommand{\wLd}{\widetilde{\Theta}_d}
\newcommand{\rc}{\rho_c}
\newcommand{\rd}{\rho_d}
\newcommand{\Lstar}{\Theta_\star}
\newcommand{\wLstar}{\widetilde{\Theta}_\star}
\newcommand{\mc}{\reals^n \times {\cal U}_c}
\newcommand{\md}{\reals^n \times {\cal U}_d}
%\newcommand{\inter}{\text{\normalfont int }}
\newcommand{\Hcl}{\HS}%{\HS_{cl}}
\newcommand{\Ccl}{C}%{C_{cl}}
\newcommand{\Fcl}{F}%{F_{cl}}
\newcommand{\Dcl}{D}%{D_{cl}}
\newcommand{\Gcl}{G}%{G_{cl}}

%%Passivity Talk Commands
\def\ba{\begin{array}}
\def\ea{\end{array}}
\def\be{\begin{equation}}
\def\ee{\end{equation}}
\def\fraz{\dst\frac}
\def\qmxr#1{\left (\begin{matrix} #1\end{matrix}\right )}
\def\qmxq#1{\left [\begin{matrix} #1\end{matrix}\right ]}
\def\qmx#1{\left (\begin{matrix} #1\end{matrix}\right )}
\def\parder#1_#2{\fraz{\partial #1}{\partial #2}}
\def\pardertr#1_#2{\fraz{\partial\tr #1}{\partial #2}}
\def\unmezzo{\fraz{1}{2}}
%%
\def\HSZero{\mathcal{H}_0}
\def\u{v}

\def\uflow{u_c}
\def\uflowTwo{\tilde{u}_c}
\def\uflowmap{u_c}
\def\uflowset{u_c}

\def\ujump{u_d}
\def\ujumpTwo{\tilde{u}_d}
\def\ujumpmap{u_d}
\def\ujumpset{u_d}

\def\yflow{y_c}
\def\yflowmap{y}
\def\yflowset{y}

\def\yjump{y_d}
\def\yjumpmap{y}
\def\yjumpset{y}

\def\restcoefficent{\varrho}
\def\restcoefficentBall{e_c}
%OLD
\def\uFlowSupply{u^\top \Gamma_c^\top}
\def\uJumpSupply{u^\top \Gamma_d^\top}

\def\yFlowSupply{\Phi_c h(x,\u)}
\def\yJumpSupply{\Phi_d h(x,\u)}
\def\SupplyOutputFlow{y^\top \Phi_c^\top \Phi_c \rho_c(y)}
\def\SupplyOutputJump{y^\top \Phi_d^\top \Phi_d \rho_d(y)}
\def\SupplyOutputFlowZeroInput{h(x,0)^\top \Phi_c^\top \Phi_c \rho_c(h(x,0))}
\def\SupplyOutputJumpZeroInput{h(x,0)^\top \Phi_d^\top \Phi_d \rho_d(h(x,0))}


%POSSIBLE NEW:
\def\uFlowSupply{u_c^\top}
\def\uJumpSupply{u_d^\top}
\def\yFlowSupply{y_c}
\def\yJumpSupply{y_d}
\def\SupplyOutputFlow{y_c^\top \rho_c(y_c)}
\def\SupplyOutputJump{y_d^\top \rho_d(y_d)}
\def\SupplyOutputFlowZeroInput{h_c(x,0)^\top \rho_c(h_c(x,0))}
\def\SupplyOutputJumpZeroInput{h_d(x,0)^\top \rho_d(h_d(x,0))}




\def\virtualInputTwo{\tilde{u}_c}
\def\virtualInputThree{\hat{u}_c}


%old F_c
\def\FCompliant{f_{c}}


%passivity based control:

\def\kappac{\kappa_c}
\def\kappad{\kappa_d}

\def\startmodif{\color{magenta}}
\def\stopmodif{\color{black}\normalcolor}
\def\ricardo#1{\color{blue}#1\normalcolor}
\def\roberto#1{\color{red}#1\normalcolor}

\DeclareMathOperator*{\argmin}{arg\,min}

% FROM JUN

\newcommand{\flow}[1]{{\color{blue}{#1}}}
\newcommand{\jump}[1]{{\color{red}{#1}}}

\newcommand{\FI}[1]{{\color{mygreen}{#1}}}
\newcommand{\chai}[1]{{\color{mygreen}{#1}}}
\newcommand{\JC}[1]{{\color{lightblue}{#1}}}
\newcommand{\jun}[1]{{\color{purple}{#1}}}

\newcommand{\ifeq}{\text{\normalfont if }}
\newcommand{\otherw}{\text{\normalfont otherwise}}
\newcommand{\inter}{\text{\normalfont int}}

\newcommand{\data}{(C,F,D,G)}
\newcommand{\z}{\x}
\newcommand{\hu}{\hat{u}}
\newcommand{\hy}{\hat{y}}
\newcommand{\hF}{\hat{F}}

\newcommand{\Nset}{\ricardo{\{1, 2, ... , N\}}}
\newcommand{\Nsety}{\{1, 2, ... , N_y\}}
\newcommand{\Nsetu}{\{1, 2, ... , N_u\}}
\newcommand{\iy}{{i_y}}
\newcommand{\iu}{{i_u}}
\newcommand{\ET}{\gamma} %  triggering event symble
\newcommand{\ETfy}{\gamma^y_\iy} % output triggering event
\newcommand{\ETfu}{\gamma^u_\iu} % input triggering event
\newcommand{\ETarg}{\xi}
\newcommand{\gfy}{g^y_{\iy}}
\newcommand{\gfu}{g^u_{\iu}}
\newcommand{\jy}{j_y}
\newcommand{\ju}{j_u}

%letters
\newcommand{\Cmc}{\mathcal{C}}
\newcommand{\Dmc}{\mathcal{D}}
\newcommand{\Lmc}{\mathcal{L}}
\newcommand{\Bmc}{\mathcal{B}}
\newcommand{\Xmc}{\mathcal{X}}

%%%%%%%%%%%%%%%%%%%%%%%%%%%%%%%%%%%%%%%%%%%%%%%%%%
% Berk's stuff
%\usepackage{tikz,circuitikz,tikz-cd,tikzscale}
%\usetikzlibrary{shapes,arrows,calc,fit}

%\usepackage{animate}

%\usepackage{pgfplots}
%\pgfplotsset{compat=newest}
%\usepgfplotslibrary{fillbetween}

%\usepgfplotslibrary{external}
%\tikzexternalize[prefix=tikz/]	%create folder named tikz

%\def\Put(#1,#2)#3{\leavevmode\makebox(0,0){\put(#1,#2){#3}}}


%Define tikz shapes
%\tikzstyle{blocknone} = [draw=none,rectangle,minimum height=1.75em,minimum width=6em]
%\tikzstyle{block} = [draw,rectangle,minimum height=1.75em,minimum width=6em]
%\tikzstyle{systemblock} = [draw,rectangle,minimum height=1.75em,minimum width=3em]
%\tikzstyle{flowblock} = [draw,rectangle,rounded corners,minimum height=3em,minimum width=20em]
%\tikzstyle{gain} = [draw,rectangle,minimum height=1em,minimum width=1em]
%\tikzstyle{sum} = [draw,circle,node distance=0.5cm]
%\tikzstyle{signal} = [coordinate]
%\tikzstyle{pinstyle} = [pin edge={to-,thin,black}]
%
%\newcommand{\Anorm}[1]{\left|#1\right|_{\cal A}}
%
%\usepackage{algorithm,algpseudocode} % \usepackage for algorithm
%
%\newcommand{\flowcost}{L_c}
%\newcommand{\jumpcost}{L_d}
%%\newcommand{\kappac}{\kappa_c}
%%\newcommand{\kappad}{\kappa_d}
%\newcommand{\alphac}{\alpha_c}
%\newcommand{\alphad}{\alpha_d}
%\newcommand{\gammac}{\gamma_c}
%\newcommand{\gammad}{\gamma_d}
%\newcommand{\totalenergy}{W}
%%%%%%%%%%%%%%%%%%%%%%%%%%%%%%%%%%%%%%%%%%%%%%%%%%

%%%%%%%%%%%%%%%%%%%%%%%%%%%%%%%%%%%%%%%%%%%%%%%%%%
% Parmita's stuff
\usepackage{calc}

\usepackage{listings}

\definecolor{mygreen}{RGB}{28,172,0} % color values Red, Green, Blue

\definecolor{mylilas}{RGB}{170,55,241}

%\lstset{language=Matlab,
%    %basicstyle=\color{red},
%    breaklines=true,
%    morekeywords={matlab2tikz},
%    keywordstyle=\color{blue},
%    morekeywords=[2]{1}, keywordstyle=[2]{\color{black}},
%    identifierstyle=\color{black},
%    stringstyle=\color{mylilas},
%    commentstyle=\color{mygreen},
%    showstringspaces=false,
%    numbers=left,
%    numberstyle={\tiny \color{black}},
%    numbersep=9pt,
%    emph=[1]{for,end,break},emphstyle=[1]\color{red},
%}

%% FROM MOHAMED
%\usepackage{bm}
%\newcommand\red[1]{{\color{red}#1}}
%\newcommand{\bsym}[1]{\boldsymbol{#1}}
%\newcommand{\mrm}[1]{\mathrm{#1}}
%\renewcommand{\mc}[1]{\mathcal{#1}}
%\newcommand{\mbb}[1]{\mathbb{#1}}
%\newcommand{\munit}[1]{[\mathrm{#1}]}
%
%% FOR HYBRID MPC
%
%\newcommand{\F}{{\cal F}}     
%\renewcommand{\L}{{\cal L}}     
%\newcommand*{\tr}{^{\mkern-1.5mu\mathsf{T}}}
%
%%Francesco's stuff
%\newcommand{\Js}{\mathcal{J}^\star}
%\newcommand{\xh}{\hat{x}}
%\newcommand{\He}{\operatorname{He}}
%\newcommand*{\QEDB}{\hfill\ensuremath{\square}}%
%\newcommand{\Sn}{\mathbb{S}^n} 
%\newcommand{\SHO}{\mathcal{S}_{\HS}}
%\newcommand{\CUD}{\overline{\C}\cup \D}
%\newcommand{\limdom}{\lim_{\substack{(t, j)\in\dom\x\\(t, j)\rightarrow\sup\dom\x}}}
%\newcommand{\limdomhat}{\lim_{\substack{(t, j)\in\dom\xh\\(t, j)\rightarrow\sup\dom\xh}}}
%\newcommand{\Cone}{\mathbf{Cone}}

%%%%%%%%%%%%%%%%%%%%%%%%%%%%%%%%%%%%%%%%%%%%%%%%%%%%%%%%%%%%%%%%%%%%%%
%%% VERSION CONTROL COMMAND
%%% For Conference, Change to "true"
%%% For Report, Change to "false"
\usepackage{ifthen}


\newboolean{Inclusion}
\setboolean{Inclusion}{false}
%\newcommand{\IfInc}[2]{\ifthenelse{\boolean{Inclusion}}{ #1 #2}{#2}}
%\newcommand{\IfIncd}[2]{\ifthenelse{\boolean{Inclusion}}{{#1}}{#2}}
\newcommand{\IfInc}[2]{\ifthenelse{\boolean{Inclusion}}{ \color{cyan}#1\color{black}#2}{#2}}
\newcommand{\IfIncd}[2]{\ifthenelse{\boolean{Inclusion}}{{\color{cyan}#1}}{#2}}

\newboolean{Personal}
\setboolean{Personal}{false}
\newcommand{\IfPers}[1]{\ifthenelse{\boolean{Personal}}{{\color{purple}#1} }{}}


\newboolean{Infinite-horizon} 
\setboolean{Infinite-horizon}{true} %False:General framework. True: Only Infinite-horizon.
\newcommand{\IfIh}[2]{\ifthenelse{\boolean{Infinite-horizon}}{#1}{{\color{purple}#2}}}

\newboolean{Two-Players} 
\setboolean{Two-Players}{true} %False:General framework. True: Only Two players formulation.
\newcommand{\IfTp}[2]{\ifthenelse{\boolean{Infinite-horizon}}{#1}{{\color{purple}#2}}}
%%%%%%%%%%%%%%%%%%%%%%%%%%



\tcbuselibrary{skins}

\newtcolorbox{mybox}[1][]{before=\centering, hbox, drop fuzzy shadow, enhanced, colback=white, rounded corners, colframe=lightblue, fonttitle=\bfseries, title=#1, center title}
\usepackage{import}
%
%\usepackage{media9}
%\usepackage{ocgx2}
\usepackage{graphicx, epstopdf}
\usepackage{graphbox}
\usepackage{animate}
%\graphicspath{{Clipart/Figures}}
\usepackage{tikz}
\usetikzlibrary{calc}
\usetikzlibrary{shadows}

\tcbuselibrary{skins,breakable}

\usepackage{bm}
\usepackage{qrcode}
\usepackage{soul}
\usepackage{macros/definitions}
% \usepackage{enumitem}

\newcommand\cmark[1][]{%
  \tikz[scale=0.4,#1]{\fill(0,.35) -- (.25,0) -- (1,.7) -- (.25,.15) -- cycle;}%
}

\newcommand\crossmark[1][]{%
  \tikz[scale=0.4,#1]{
    \fill(0,0)--(0.1,0) .. controls (0.5,0.4) .. (1,0.7)--(0.9,0.7) ..  controls (0.5,0.5) ..(0,0.1) --cycle;
    \fill(1,0.1)--(0.9,0.1) .. controls (0.5,0.3) .. (0,0.7)--(0.1,0.7) .. controls (0.5,0.4) ..(1,0.2) --cycle;
  }%
}

\newcommand{\imgwithmask}[6][white]{%
  \begin{tikzpicture}
    % place the image; width adapts to the surrounding box (\linewidth)
    \node[inner sep=0, anchor=north west] (im) at (0,0)
         {\includegraphics[width=\linewidth]{#2}};
    % clip to the image bounds so the mask never spills outside
    \begin{scope}
      \clip (im.south west) rectangle (im.north east);
      % rectangle: start at (top-left + UL-x right, UL-y down), then width x height
      \fill[#1] ($(im.north west)+(#3,-#4)$) rectangle ++(#5,-#6);
    \end{scope}
  \end{tikzpicture}%
}

\setbeamertemplate{theorem}[numbered]                % use numbered theorems
\setbeamerfont{theorem name}{series=\bfseries}       % make “Theorem” bold

% Colors for corrections
\newcommand{\pn}[1]{{\color{red}#1}}
\newcommand{\pnn}[1]{{\color{violet} #1}}
\newcommand{\pno}[1]{{\color{blue} #1}}
\newcommand{\mathcolorbox}[2]{\colorbox{#1}{$\displaystyle #2$}}
\newcommand{\hlc}[2][yellow]{{%
    \colorlet{foo}{#1}%
    \sethlcolor{foo}\hl{#2}}%
}

% \usepackage[backend=bibtex,style=authoryear]{biblatex}
% \addbibresource{biblio.bib} % <-- your .bib

% Beamer footnote w/o rule (do NOT load footmisc)
\setbeamertemplate{footnote}{\parindent0pt\tiny\insertfootnotemark\ \insertfootnotetext\par}

% Overlay-aware footnote citation
\newcommand{\shortfootcite}[2][1-]{\footnote<#1>{\textcite{#2}}}



\newcommand{\phantombox}[2]{\only<#1>{\vphantom{#2}}}

\newcommand\blfootnote[1]{%
  \begingroup
  \renewcommand\thefootnote{}\footnote{#1}%
  \addtocounter{footnote}{-1}%
  \endgroup

\makeatletter
}

\usepackage{ragged2e}
\usepackage[hang,flushmargin]{footmisc}
\usepackage{xparse}
% Define colors
\definecolor{flows}{RGB}{41, 41, 98}
\definecolor{jumps}{RGB}{238, 20, 30}
\definecolor{epsC}{RGB}{67, 67, 205} % epsilon-net for (C \cap \cal U) \setminus \cal A
\definecolor{epsD}{RGB}{243, 17, 26} % epsilon-net for (D \cap \cal U) \setminus \cal A

\def\Put(#1,#2)#3{\leavevmode\makebox(0,0){\put(#1,#2){#3}}}

% Remove prefix "Figure" in figures
\setbeamertemplate{caption}{\raggedright\insertcaption\par}

% ============================================
% THEOREM ENVIRONMENTS
% ============================================

% --- Base brand colors (dark, good for title bars with white text) ---
\definecolor{myblue}{HTML}{0B3D91}   % blue 600
\definecolor{mygreen}{HTML}{047857}  % green 700
\definecolor{myorange}{HTML}{B45309} % orange/brown 700
\definecolor{mygray}{HTML}{334155}   % slate 700

% Undefine the existing environments completely
\makeatletter
\let\definition\@undefined
\let\enddefinition\@undefined

\let\proposition\@undefined
\let\endproposition\@undefined

\let\corollary\@undefined
\let\endcorollary\@undefined

\let\theorem\@undefined
\let\endtheorem\@undefined
\let\lemma\@undefined
\let\endlemma\@undefined
\makeatother

% Single shared counter for all theorem types
\newcounter{thmcounter}
\renewcommand{\thethmcounter}{\arabic{thmcounter}}

% Helper function to create theorem environments
% #1 = additional tcolorbox options (optional)
% #2 = environment name (e.g., definition, theorem)
% #3 = display name (e.g., Definition, Theorem)  
% #4 = color (e.g., myblue, mygreen)

% Method 1: Use overlay-aware counter
\newcommand{\createtheoremenv}[4][]{%
    \newenvironment{#2}[1][]{%
        % Only increment counter on first overlay
        \only<1>{\refstepcounter{thmcounter}}%
        % Set itemize colors to match environment
        \setbeamercolor{itemize item}{fg=#4}%
        \setbeamercolor{itemize subitem}{fg=#4!70}%
        \setbeamertemplate{itemize item}{\color{#4}$\blacktriangleright$}%
        \setbeamertemplate{itemize subitem}{\color{#4!70}$\triangleright$}%
        % Set enumerate colors
        \setbeamercolor{enumerate item}{fg=#4}%
        \setbeamercolor{enumerate subitem}{fg=#4!70}%
        \setbeamertemplate{enumerate item}{\color{#4}\insertenumlabel.}%
        \setbeamertemplate{enumerate subitem}{\color{#4!70}(\insertsubenumlabel)}%
        % Check if optional title is provided
        \ifx\relax##1\relax
        \begin{tcolorbox}[
            enhanced,
            breakable,
            colback=#4!5,
            colframe=#4,
            colbacktitle=#4,
            coltitle=white,
            attach boxed title to top left={yshift=-2mm,xshift=2mm},
            boxed title style={sharp corners},
            arc=1mm,
            boxrule=1pt,
            #1, % Additional options
            title={#3 %\thethmcounter
            }
        ]
        \else
        \begin{tcolorbox}[
                enhanced,
                breakable,
                colback=#4!5,
                colframe=#4,
                colbacktitle=#4,
                coltitle=white,
                attach boxed title to top left={yshift=-2mm,xshift=2mm},
                boxed title style={sharp corners,text width=0.9\textwidth},
                arc=1mm,
                boxrule=1pt,
                #1, % Additional options
                title={#3: %\thethmcounter: 
                ##1}
            ]
            \fi
            }{%
        \end{tcolorbox}%
    }%
}

% Create all theorem environments using the helper function
\createtheoremenv{definition}{Definition}{mygray}
\createtheoremenv{theorem}{Theorem}{myblue}
\createtheoremenv{lemma}{Lemma}{myblue}
\createtheoremenv{proposition}{Proposition}{myblue}
\createtheoremenv{corollary}{Corollary}{myblue}
% \createtheoremenv{assumption}{Assumption}{myorange}
\createtheoremenv{remark}{Remark}{mygray}

% current overlay = normal; otherwise = gray
\newcommand{\fadepast}[1]{\alt<.>{#1}{\textcolor{gray}{#1}}}
\newcommand{\fadeat}[2]{\alt<#1>{#2}{\textcolor{gray}{#2}}}

\newcommand{\disponslide}[2]{%
  \alt<#1>{#2}{\phantom{#2}}}

\begin{document}


\NewDocumentEnvironment{mydef}{o m}{%
  \par\medskip%
  \noindent
  \textbf{Definition~#2.}%            ← bold “Definition” and your number
  \IfValueT{#1}{\ (\textit{#1}).}%    ← optional title in italics  
  \quad
}{%
  \par\medskip
}

\NewDocumentEnvironment{mythm}{o m}{%
  \par\medskip%
  \noindent
  \textbf{Theorem~#2.}%            ← bold “Theorem” and your number
  \IfValueT{#1}{\ (\textit{#1}).}%    ← optional title in italics  
  \quad
}{%
  \par\medskip
}

%% BEGIN WORKSHOPCDC19/BERKALTIN/SLIDES/TITLEPAGE-WELCOME_IFACWC20 %%
%% BEGIN WORKSHOPCDC16/TITLEPAGE-DYNAMICPROPERTIES %%
% !TEX TS-program = personallatex

%Analysis of Interconnections of Hybrid Systems: an Input/Output Approach
\title{}%Interconnections and Control of Hybrid Systems}
\subtitle{\bf 
Control of Uncertain Hybrid Systems
\vspace{0.7in}}
%Control and Interconnections of Hybrid Systems\vspace{0.7in}}
%\subtitle{\Large of Hybrid Systems \vspace{1in}}
\author{\vspace{-0.35in}
\bf Carlos A. Montenegro G.}
\institute{
%\vspace{-0.05in}
%$\star$ 
University of California, Santa Cruz, USA\\
%\vspace{0.1in}{\it Joint work with Andrew R. Teel and Rafal Goebel}
%\\
\vspace{0.05in}

{
\begin{center}

-----
\\
\it Qualifying Exam for Advancement to PhD Candidacy
\\
\medskip
August 11, 2025
\\
\medskip
\tiny Committee: Gabriel H. Elkaim (UCSC), Ricardo G. Sanfelice (UCSC), Lars Lindemann (USC/ETH Zurich), Murat Arcak (UCB)
\end{center}
}

%{\it \tiny Collaborators: A. R. Teel (UCSB), E. Frazzoli (MIT), R. Goebel (LUC), C. Prieur (LAAS), L. Praly (E. Mines)}
}

% Temporarily out
%\addlogo{\ucsblogo}
%\addlogo{\mitlogo}
%\addlogo{\ensmplogo}

% Use this line to get some text in the logo bar as well
\logobartext{Montenegro G. - UCSC
- \insertframenumber/\inserttotalframenumber
%- \today\  
%- \insertframenumber/\inserttotalframenumber\strut\quad
}
%\logobartext{}
\setbluesheetcolors

%\begin{figure}
% \includegraphics[scale=0.14]{Clipart/Figures/MissileInterception} 
% \caption{Iron Dome. Anas Baba/AFP/Getty Images
%}
%  \label{fig:IronD}
% \end{figure}
%
%\begin{columns}[T]
%\begin{column}{0.5\textwidth}
%\end{column}
%\begin{column}{0.5\textwidth}
%\end{column}
%\end{columns}
%%%%%%%%%%%%%%%%%%%%%%%%%%%%%%%%%%%%\input{Slides/
% Intro - Motivation
% \input{Slides/TitlePage}\setwhitesheetcolors 
% \input{Slides/Introduction-Motivation}
% \input{Slides/Introduction-UncertainSystems}
% \input{Slides/Introduction-LearningCertificates}

% % Prelims - Hybrid Systems
% \input{Slides/Preliminaries-ModelingHybridDynamics}
% \input{Slides/Preliminaries-HybridEquation}
% \input{Slides/Preliminaries-HyConnections2Frameworks}

% % Thrust 1 - Probabilistic Learning-Based Modelling and Control
% {
%   \setbeamertemplate{headline}{}
%   \setbeamertemplate{footline}{}
%   \setbeamertemplate{navigation symbols}{}
%   \setbeamertemplate{background canvas}[default] % no theme bg image
%   \setbeamercolor{background canvas}{bg=white}   % force pure white

%   \begin{frame}[plain,noframenumbering]
%     \vspace{2cm}
%     \begin{minipage}[t][8cm][t]{\textwidth}
%       $\boxed{\textup{\textbf{Thrust 1}: Probabilistic Learning-Based Modeling and Control}}$

%       % \vspace{0.5cm}
%       \only<1->{
%         \[
%           \begin{split}
%                   &\calH_P \>:\> \left\{ 
%               \begin{matrix}
%               \hspace{0.2cm} \dot{x} \in \pno{F_P}(x,\pno{u_C}) := \left\{ f(x, \pno{u_C}) + \pno{d_C}(x, \pno{u_C})  \right\}
%                 \quad & (x,\pno{u_C}) \in \pno{C_P}
%               \\
%               x^+ \in \pn{G_P}(x,\pn{u_D}) := \left\{ g(x, \pn{u_D}) + \pn{d_D}(x, \pn{u_D})  \right\}
%               \quad & \hspace{0.15cm} (x,\pn{u_D}) \in \pn{D_P}
%           \end{matrix}
%           \right. 
%           \end{split}
%         \]


%         The maps $\pno{d_C}$ and $\pn{d_D}$ are \textbf{unknown}.
%       }

%       \vspace{0.5cm}
      
%       {\small
%         \only<2->{
%           \textbf{Tasks}:
%           \setbeamercovered{transparent}
%           \begin{itemize}
%             \item<3-> \fadeat{3}{Develop \textbf{multivariate GPs} to learn a surrogate $\widehat d_\star$.}
%             \item<4-> \fadeat{4}{Design \textbf{probabilistic error bounds} for such GPs.}
%             \item<5-> \fadeat{5}{Design of \textbf{stabilizing feedback laws}.}
%           \end{itemize}
%         }
%       }
%     \end{minipage}
%   \end{frame}
% }
% \input{Slides/Preliminaries-IntroGPs}
% \input{Slides/Preliminaries-GPsIntuition}
% \input{Slides/Preliminaries-MVGPDefinition}
% \input{Slides/Results-RegularityGP}
% \input{Slides/Results-PosteriorGP}
% \input{Slides/Results-ErrorBoundsGPs}
% \input{Slides/Results-GPsSummary}
% \input{Slides/FutureWork-GPs}


% % Thrust 2 - Learning-Based Safety Control for Uncertain Hybrid Systems
% {
%   \setbeamertemplate{headline}{}
%   \setbeamertemplate{footline}{}
%   \setbeamertemplate{navigation symbols}{}
%   \setbeamertemplate{background canvas}[default] % no theme bg image
%   \setbeamercolor{background canvas}{bg=white}   % force pure white

%   \begin{frame}[plain,noframenumbering]
%     \vspace{0.2cm}
%     \begin{minipage}[t][8cm][t]{\textwidth}
%       $\boxed{\textup{\textbf{Thrust 2}: Learning-Based Safety Control for Uncertain Hybrid Systems}}$

%       % \vspace{0.5cm}
%       \only<1->{
%         \[
%           \begin{split}
%                   &\calH_P \>:\> \left\{ 
%               \begin{matrix}
%               \hspace{0.2cm} \dot{x} \in \pno{F_P}(x,\pno{u_C}) %:= \left\{ f(x, \pno{u_C}) + \pno{d_C(x, \pno{u_C})}  \right\}
%                 \quad & (x,\pno{u_C}) \in \pno{C_P}
%               \\
%               x^+ \in \pn{G_P}(x,\pn{u_D}) % := \left\{ g(x, u_C) + \pn{d_D(x, \pn{u_D})}  \right\}
%               \quad & \hspace{0.15cm} (x,\pn{u_D}) \in \pn{D_P}
%           \end{matrix}
%           \right. 
%           \end{split}
%         \]
%       }

%       \vspace{0.2cm}
      
%       {\small
%         \uncover<2->{
%           \textbf{Tasks}:
%           \setbeamercovered{transparent}
%           \begin{itemize}
%             \item<3-> \fadeat{3}{Design of \textbf{vector-valued CBFs} for hybrid inclusions.}
%             \item<4-> \fadeat{4}{Study the \textbf{continuity} of \textbf{optimization-based controllers} using vector-valued CBFs.}
%             \item<5-> \fadeat{5}{Study the \textbf{nominal robustness} under \textbf{discontinuous} safeguarding controllers.}
%           \end{itemize}
%         }
%         \vspace{0.4cm}
%       }

%       {\footnotesize
%         \only<6>{
%           \textbf{Publications}:

%           \begin{itemize}
%             \item \textbf{C. A. Montenegro G.}, H. M. Sweatland, K. Currier, W. E. Dixon, and R. G. Sanfelice, ``\emph{Pointwise Optimal Feedback Laws for Hybrid Inclusions using Multiple Control Barrier Functions}," Proceedings of the IEEE Conference on Decision and Control, 2025.
%           \end{itemize}
%         }
%       }
%     \end{minipage}
%   \end{frame}
% }
% \input{Slides/Introduction-mCBFsMotivation}
% \input{Slides/Preliminaries-FpI}
% \input{Slides/Preliminaries-pAS}
% \input{Slides/Preliminaries-mBFs}
% \input{Slides/Preliminaries-mCBFs}
% \input{Slides/Results-FpI}
% \input{Slides/Results-LpASmCBFs}
% \input{Slides/Results-Continuity_PointwiseOptimalMaps}
% \input{Slides/Results-RobustnessKrasovskii}
% \input{Slides/Example-SimulationsmCBFs}
% \input{Slides/FutureWork-mCBFs}


% % Thrust 3 - Learning-Based Certificates for Asymptotic Stability, Cost Evaluation, and Invariance in Uncertain Hybrid Systems
% {
%   \setbeamertemplate{headline}{}
%   \setbeamertemplate{footline}{}
%   \setbeamertemplate{navigation symbols}{}
%   \setbeamertemplate{background canvas}[default] % no theme bg image
%   \setbeamercolor{background canvas}{bg=white}   % force pure white

%   \begin{frame}[plain,noframenumbering]
%     % \vspace{1cm}
%     \begin{minipage}[t][8cm][t]{\textwidth}
%       $\boxed{\textup{\textbf{Thrust 3}: Learning-Based Certificates for  Hybrid Systems}}$

%       % \vspace{0.5cm}
%       \only<1->{
%         \[
%           \begin{split}
%                   &\mathcal{H} \>:\> \left\{ 
%               \begin{matrix}
%               \hspace{0.2cm} \dot{x} \in \pno{F}(x)
%                 \quad & x \in \pno{C}
%               \\
%               x^+ \in \pn{G}(x)
%               \quad & \hspace{0.15cm} x \in \pn{D}
%           \end{matrix}
%           \right. 
%           \end{split}
%         \]
%       }

%       \vspace{0.5cm}
      
%       {\small
%         \only<2->{
%           \textbf{Tasks}:
%         \setbeamercovered{transparent}
%         \begin{itemize}
%           \item<3-> \fadeat{3}{Design a learning-based surrogate \textbf{Lyapunov function} for hybrid systems.}
%           \item<4-> \fadeat{4}{Design learning-based surrogates to \textbf{upper bound} a \text{cost} for hybrid systems.}
%           \item<5-> \fadeat{5}{Show that the learning-based \textbf{surrogate to upper bound the cost} is also a \textbf{Lyapunov} function.}
%         \end{itemize}
%         }
%       }

%       {\footnotesize
%         \only<6>{

%           \begin{tikzpicture}[remember picture,overlay]
%                 \fill[white,opacity=.9] % or black,opacity=.30 to dim
%                     (current page.south west) rectangle (current page.north east);

%                     % % the box on top (centered)
%                     \node[anchor=center, xshift=0cm, yshift=-2cm] at (current page.center){%
%                     \begin{minipage}{0.9\paperwidth} % choose width
%                         \textbf{Publications}:

%                         \begin{itemize}
%                           \item \textbf{C. A. Montenegro G.}, S. J. Leudo, and R. G. Sanfelice, “\emph{A Data-Driven Approach for Certifying Asymptotic Stability and Cost Evaluation for Hybrid Systems},” in Proceedings of the 27th ACM International Conference on Hybrid Systems: Computation and Control, 2024.
%                           \item \textbf{C. A. Montenegro G.}, S. J. Leudo, and R. G. Sanfelice, “\emph{Learning Certificates for Asymptotic Stability and Cost Evaluation in Hybrid Dynamical Systems},” To appear in Nonlinear Analysis: Hybrid Systems, 2025.
%                         \end{itemize}
%                                   \end{minipage}%
%                                   };
%                           \end{tikzpicture}
%         }
%       }
%     \end{minipage}
%   \end{frame}
% }
% \input{Slides/Results-LyapunovNN}
% \input{Slides/Preliminaries-EpsilonNets}
% \input{Slides/Introduction-MotivationOptimizationLyapunov}
% \input{Slides/Results-ExtendingLyapCondfromSamples_Main}

% % Prelims - Motivation Inverse Optimality
% \input{Slides/Introduction-MotivationInverseOptimality}

% % Outline
% \input{Slides/Introduction-Outline}

% % ISSf for Hybrid Systems
% \input{Slides/Preliminaries-ISSfDefinition}
% \input{Slides/Preliminaries-ISSfBF}
% \input{Slides/Results-ISSfUsingBF}
% \input{Slides/Introduction-ProblemStatement}
% \input{Slides/Results-ISSfCBFs}

% % % Inverse optimality
% % \input{Slides/Introduction-ZeroSumHybridGame}
% \input{Slides/Results-IO_SafetyFilters}

% % Numerical Example
% \input{Slides/Example-OscillatorWImpacts}

% % Funding
% \input{Slides/Funding}

% \begin{frame}{\bf Acknowledgements}
%     This research has been partially supported by
%     \begin{itemize}
%         \item the National Science Foundation under Grant no. CNS-2039054 and Grant no. CNS-2111688,
%         \item the Air Force Office of Scientific Research under Grant no. FA9550-19-1-0169, Grant no. FA9550-20-1-0238, Grant no. FA9550-23-1-0145, and Grant no. FA9550-23-1-0313,
%         \item the Air Force Research Laboratory under Grant no. FA8651-22-1-0017 and Grant no. FA8651-23-1-0004,
%         \item the Army Research Office under Grant no. W911NF-20-1-0253,
%         \item and the Department of Defense Grant no. W911NF-23-1-0158.
%     \end{itemize}
% \end{frame}


\end{document}